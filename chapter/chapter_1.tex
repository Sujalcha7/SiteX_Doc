\chapter{Introduction}
\pagenumbering{arabic}
\section{Background Introduction}

Anyone who walks down a busy street in Kathmandu, Patan, or Bhaktapur can see the cycle of hope and hardship firsthand. A new, brightly lit cafe opens, full of promise and excitement. A few months later, that same spot might have a "To-Let" sign hanging in its window. For every successful business that becomes a local landmark, many others quietly disappear, often taking an entrepreneur's life savings with them. This isn't just a business problem; it's a deeply human one that affects families and communities across Nepal.

The age-old wisdom has always been that success in retail or hospitality comes down to three things: "location, location, location." But for most small business owners in our country, choosing a location has always been a matter of intuition. Decisions are often based on gut feeling, family advice, or simply what feels like a "good area." While this experience is valuable, it's also a huge gamble. It's a process filled with uncertainty, where a single wrong choice can be the difference between a thriving business and a failed dream.

As computer science students, we are taught about the power of data. We learn how global giants like Amazon or Starbucks use sophisticated data analysis, Geographic Information Systems (GIS), and Artificial Intelligence (AI) to make these exact decisions with incredible accuracy. They don't guess; they forecast. They analyze dozens of factors—from foot traffic and competitor density to local income levels and demographic profiles—to predict a location's potential before investing a single rupee. This creates a huge analytical gap between them and the entrepreneurs we see every day.

We saw this gap as the perfect challenge for our final year project. We asked ourselves: can we use the skills we've learned in our engineering course to build a tool that levels the playing field? Can we create an intelligent system that brings the power of predictive analytics to a business owner in Bhaktapur, helping them make a smarter, safer investment? This project is our attempt to answer that question.

\section{Motivation}

Our motivation for this project is both personal and academic. Personally, as students living in Nepal, we see this problem all around us. We are driven by the desire to build something that has a real, tangible impact on our local community. The idea that our technical skills could help a fellow Nepali make a more confident business decision is a powerful motivator. We want to build a tool that we would be proud to show a friend or family member who is thinking of opening their own shop.

Academically, this project presents the perfect challenge. It forces us to go beyond textbook theory and tackle a complete, end-to-end data science problem. We are motivated by the opportunity to apply what we've learned about GIS, data engineering, and machine learning to a messy, real-world scenario. The process of building our own unique dataset from scratch—incorporating spatial, business, and demographic data—training predictive models, and deploying a feature-rich web application is an incredible learning experience that encapsulates everything we've studied.

\section{Statement of Problem}

The core problem is this: opening a new business in Nepal is a high-stakes gamble, primarily because entrepreneurs lack accessible tools to conduct comprehensive market and site analysis.This reliance on intuition over data leads to a high risk of choosing a suboptimal location, resulting in financial loss and business failure. Our project directly addresses the lack of an integrated, quantitative decision-making platform. We aim to solve the problem of uncertainty by providing a system that delivers a clear, evidence-based forecast of a location's suitability, tailored to the user's specific business goals.

\section{Main Objective}

The principal objective of this project is:
\begin{itemize}
	\item {To design, develop, and deploy an intelligent, data-driven system integrating predictive analytics and user-defined requirements for personalized site selection, supporting decision-making.}
\end{itemize}

\section{Scopes and Application}

\subsection{Scopes}

To ensure the successful delivery of a robust and feature-rich platform, we have defined the following scope:
\begin{itemize}
	\item \textbf{Where?} Our project will focus exclusively on the core commercial area of \textbf{Bhaktapur}.
	\item \textbf{What kind of shop?} Our initial model will be an expert on one thing: \textbf{cafes and coffee shops}, with a system designed to be scalable to other retail types in the future.
	\item \textbf{What data?} Our system will be built on a multi-faceted dataset including:
	      \begin{itemize}
		      \item Publicly available geospatial data from OpenStreetMap.
		      \item Manually collected "ground-truth" data (e.g., business verification, visibility scores).
		      \item Demographic proxy data, derived from scraped real estate values and business density analysis.
		      \item Crowdsourced property listings for available "To-Let" spaces.
	      \end{itemize}
	\item \textbf{What it is?} This tool is a comprehensive \textbf{Location Intelligence Platform}. It combines predictive analytics with a real estate listing feature to provide an end-to-end solution for entrepreneurs.
\end{itemize}

\subsection{Application}

The finished application is designed to be a strategic partner for entrepreneurs and planners. Here’s how it will be used:
\begin{itemize}
    \item \textbf{Personalized Market Discovery:} Users can input their business profile (e.g., 'Budget Student Cafe'), budget, and space requirements to generate a personalized "Opportunity Zone" heatmap, guiding them to areas that best fit their strategy.
    \item \textbf{Integrated Property Search:} Users can discover and filter available "To-Let" properties directly on the map, bridging the gap between analysis and action.
    \item \textbf{Deep Site Analysis:} For any point or property, users can get a detailed suitability score, a breakdown of influencing factors, and an analysis of the local competition and trade area.
    \item \textbf{Strategic Comparison:} The platform allows users to save and compare multiple locations side-by-side to make a final, evidence-based decision.
    \item \textbf{A Tool for Planners:} Urban planners and real estate agents can use the rich data layers (competition density, demographic proxies) to gain a deeper understanding of the city's commercial ecosystem.
\end{itemize}