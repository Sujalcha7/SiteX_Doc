\chapter{Introduction}
\pagenumbering{arabic}
\section{Background Introduction}

Anyone who walks down a busy street in Kathmandu, Patan, or Bhaktapur can see the cycle of hope and hardship firsthand. A new, brightly lit cafe opens, full of promise and excitement. A few months later, that same spot might have a "To-Let" sign hanging in its window. For every successful business that becomes a local landmark, many others quietly disappear, often taking an entrepreneur's life savings with them. This isn't just a business problem; it's a deeply human one that affects families and communities across Nepal.

The age-old wisdom has always been that success in retail or hospitality comes down to three things: "location, location, location." But for most small business owners in our country, choosing a location has always been a matter of intuition. Decisions are often based on gut feeling, family advice, or simply what feels like a "good area." While this experience is valuable, it's also a huge gamble. It's a process filled with uncertainty, where a single wrong choice can be the difference between a thriving business and a failed dream.

As computer science students, we are taught about the power of data. We learn how global giants like Amazon or Starbucks use sophisticated data analysis, Geographic Information Systems (GIS), and Artificial Intelligence (AI) to make these exact decisions with incredible accuracy. They don't guess; they forecast. They analyze dozens of factors—from foot traffic and competitor density to local income levels—to predict a location's potential before investing a single rupee. This creates a huge analytical gap between them and the local entrepreneurs we see every day.

We saw this gap as the perfect challenge for our final year project. We asked ourselves: can we use the skills we've learned in our engineering course to build a tool that levels the playing field? Can we create an intelligent system that brings the power of predictive analytics to a local business owner in Bhaktapur, helping them make a smarter, safer investment? This project is our attempt to answer that question.

\section{Motivation}

Our motivation for this project is both personal and academic. Personally, as students living in Nepal, we see this problem all around us. We are driven by the desire to build something that has a real, tangible impact on our local community. The idea that our technical skills could help a fellow Nepali make a more confident business decision is a powerful motivator. We want to build a tool that we would be proud to show a friend or family member who is thinking of opening their own shop.

Academically, this project presents the perfect challenge. It forces us to go beyond textbook theory and tackle a complete, end-to-end data science problem. We are motivated by the opportunity to apply what we've learned about GIS, data engineering, and machine learning to a messy, real-world scenario. The process of building our own unique dataset from scratch, training a predictive model like XGBoost, and deploying it as a live web application is an incredible learning experience that encapsulates everything we've studied.

\section{Statement of Problem}

The core problem is this: opening a new business in Nepal is a huge gamble, primarily because there is no accessible, data-driven way for a local entrepreneur to assess the potential of a location. This reliance on intuition over analysis leads to a high risk of choosing the wrong spot, resulting in financial loss and business failure. Our project directly addresses this by tackling the lack of a quantitative decision-making tool. We aim to solve the problem of uncertainty in site selection by providing a system that delivers a clear, evidence-based forecast of a location's viability.

\section{Main Objectives}

The principal objective of this project is:
\begin{itemize}
	\item {To design, develop, and deploy an intelligent, data-driven system that accurately predicts the potential success of a new branch/store in a site.}
\end{itemize}

\section{Scopes and Application}

\subsection{Scopes}

To make sure we can successfully complete this project within our academic timeline, we have defined a clear and focused scope:
\begin{itemize}
	\item \textbf{Where?} Our project will focus exclusively on the core commercial area of \textbf{Bhaktapur}.
	\item \textbf{What kind of shop?} Our first model will be an expert on one thing: \textbf{cafes and coffee shops}.
	\item \textbf{What data?} We will only use data that is publicly available (like from OpenStreetMap) or that we can collect ourselves manually.
	\item \textbf{What it isn't?} This tool is for predicting a location's potential. It is not a real estate listing site or a business management software.
\end{itemize}

\subsection{Application}

The finished application is designed to be a practical tool for real people. Here’s how it could be used:
\begin{itemize}
	\item \textbf{Help an Entrepreneur Before They Invest:} A user can check a location's potential score *before* they sign an expensive rental agreement.
	\item \textbf{Compare Two or More Locations:} If someone is trying to decide between a spot in Durbar Square and another in Taumadhi, our tool can provide an objective, data-driven comparison.
	\item \textbf{Understand the Market:} Users can click around different neighborhoods to understand why certain areas are hotspots and others are not, helping them with their market research.
	\item \textbf{A Tool for Others:} Beyond entrepreneurs, this system could also be useful for urban planners or real estate agents trying to understand the commercial landscape of a city.
\end{itemize}