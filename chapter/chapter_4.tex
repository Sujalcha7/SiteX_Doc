\chapter{Feasibility Study}

A feasibility study is conducted to evaluate the practicality, viability, and suitability of the proposed system before full-scale development. This chapter analyzes the feasibility of the \textbf{Predictive Site Selection System} under five major aspects: technical feasibility, operational feasibility, economic feasibility, legal and ethical feasibility, and scheduling feasibility.

\section{Technical Feasibility}

The proposed system is technically feasible as it is based on established and widely adopted technologies that align with the academic curriculum of the Bachelor of Computer Engineering program. The system integrates Geographic Information Systems (GIS), data engineering techniques, and machine learning algorithms to perform predictive location analysis.

Publicly available geospatial data from Google Maps, along with manually collected and ethically scraped datasets, provide sufficient input for analysis. Machine learning models such as regression and classification techniques can be used to generate suitability scores based on multiple influencing factors. The backend of the system can be implemented using Python-based frameworks, while spatial processing can be handled using open-source libraries such as GeoPandas and scikit-learn (BallTree/KDTree) for real-time spatial indexing.

All required tools, frameworks, and libraries are open-source and supported by extensive documentation, eliminating the need for proprietary software or specialized hardware. Hence, the system can be developed and deployed within the available technical resources.

\section{Operational Feasibility}

The system is operationally feasible as it is designed to be user-friendly and accessible to non-technical users such as small business owners, entrepreneurs, urban planners, and real estate agents. The platform presents complex analytical results through visual elements such as interactive maps, suitability scores, and comparative charts, making interpretation straightforward.

The system functions as a decision-support tool rather than an automated decision-maker, allowing users to retain control over their final business decisions. By integrating market analysis with location insights in a single platform, the system fits naturally into existing business planning workflows and encourages real-world adoption.

\section{Economic Feasibility}

The project is economically feasible, particularly as an academic and prototype-level system. Development costs are minimal since the system utilizes free and open-source software, publicly available datasets, and standard computing infrastructure.

No licensing fees are required for development tools or GIS data. The primary investment involves time and effort from the project team, which is already allocated as part of the semester project. From a user perspective, the system offers significant economic benefits by reducing the risk of selecting unsuitable business locations and minimizing potential financial losses.

\section{Legal and Ethical Feasibility}

The system is legally and ethically feasible as it relies on publicly available and ethically sourced data. No sensitive personal data is collected, stored, or processed. All scraped data is used in aggregate form and solely for academic and analytical purposes.

Ethically, the system promotes fairness by providing data-driven insights to small and local entrepreneurs who may otherwise lack access to advanced analytical tools. Transparency is maintained by clearly explaining suitability factors and avoiding opaque decision-making processes. The system provides recommendations rather than enforcing decisions, ensuring ethical responsibility remains with the user.

\section{Scheduling Feasibility}

The project is scheduling feasible within the academic timeframe of the seventh semester. The development process is divided into manageable phases including requirement analysis, data collection, preprocessing, model development, system integration, testing, and documentation.

By limiting the project scope to the core commercial area of Bhaktapur and focusing on a single business type (cafes), complexity is controlled. Parallel task distribution among team members further supports timely completion. With proper planning and regular supervision, the project can be successfully completed within the given schedule.
