\chapter{PROJECT MANAGEMENT}
\section{Team Members}
\text{This project is the joint effort of: }
\begin{enumerate}
	\item Anish Maka      - 780304
	\item Roj Gosai       - 780334
	\item Subekshya Kadel - 780341
	\item Sujal Koju      - 780342
\end{enumerate}
% \section{Feasibility Study}
\section{Work Breakdown Structure}

\begin{table}[h]
\centering
\setlength{\tabcolsep}{5pt}  % Decreased horizontal padding
\renewcommand{\arraystretch}{1.5}  % Increases vertical padding
\begin{tabular}{|c|l|c|c|c|c|c|c|}
\hline
S.N & \multirow{2}{*}{Job Description} & \multicolumn{6}{c|}{Month} \\
\cline{3-8}
 & & 1st & 2nd & 3rd & 4th & 5th & 6th \\
 & & month & month & month & month & month & month \\
\hline
1. & Project Identification & \cellcolor{gray!40} & & & & & \\
\hline
2. & Analysis & \cellcolor{gray!40} & \cellcolor{gray!40} & & & & \\
\hline
3. & Design & & \cellcolor{gray!40} & \cellcolor{gray!40} & \cellcolor{gray!40} & & \\
\hline
4. & Coding \& testing & & & \cellcolor{gray!40} & \cellcolor{gray!40} & \cellcolor{gray!40} & \\
\hline
5. & Implementation & & & & & \cellcolor{gray!40} & \cellcolor{gray!40} \\
\hline
6. & Documentation & \multicolumn{6}{c|}{\cellcolor{gray!40}} \\
\hline
\end{tabular}
\caption{Project Timeline}
\label{tab:project-timeline}
\end{table}

\section{Software Requirements}

The software requirements for the development and deployment of the predictive site selection system include:
\begin{enumerate}
    \item \textbf{Collaboration \& Planning:} Google Drive, Trello, Discord
    \item \textbf{Backend Development:} Python 3.9+, \textbf{FastAPI} (for building a high-performance API), \textbf{Uvicorn} (as the ASGI server).
    \item \textbf{Machine Learning \& Spatial Analysis:} Scikit-learn, XGBoost, Pandas, Numpy, NetworkX (for graph-based routing), and GeoPandas (for offline spatial analysis).
    \item \textbf{Data Storage:} CSV/GeoJSON (File-based storage).
    \item \textbf{GIS Tools:} QGIS Desktop
    \item \textbf{Frontend Development:} Node.js, React.js, \textbf{Vite} (build tool), \textbf{Tailwind CSS} (styling).
    \item \textbf{Mapping Library:} \textbf{Leaflet} (via React-Leaflet).
    \item \textbf{Routing Engine:} Custom local isochrone and path generation using \textbf{NetworkX} and local road network data (replacing external APIs).
    \item \textbf{Version Control:} Git, GitHub.
    \item \textbf{Code Editor:} Visual Studio Code.
    \item \textbf{Documentation:} MiKTeX in VS Code.
\end{enumerate}


\section{Hardware Requirements}

The hardware requirements for the development machine and end-user system are as follows:
\begin{enumerate}[label=\alph*.]
	\item \textbf{Development Machine:}
	      \begin{itemize}
		      \item \textbf{RAM:} Minimum 8GB, recommended 16GB (for handling datasets and running the database server).
		      \item \textbf{Processor:} Modern multi-core CPU (e.g., Intel Core i5 / AMD Ryzen 5 or better).
		      \item \textbf{Operating System:} Windows, macOS, or Linux.
	      \end{itemize}
	\item \textbf{End-User System:}
	      \begin{itemize}
		      \item A modern web browser (e.g., Chrome, Firefox, Safari) on any desktop or mobile device.
	      \end{itemize}
\end{enumerate}

\subsection{Functional Requirements}
The functional requirements define the specific behaviors and capabilities of the SiteCortex platform. The system must:

\begin{enumerate}

    \item[\textbf{a.}] \textbf{Location Selection via Map Interface:} Provide an interactive map interface that allows users to select and evaluate a potential café or restaurant location by pin-pointing a specific latitude and longitude.
    \item[\textbf{b.}] \textbf{Map Visualization:} Display a color-coded suitability map of Bhaktapur, highlighting high-potential areas tailored to the user's specific business focus.
    \item[\textbf{c.}] \textbf{Integrated Property Discovery:} Display icons for currently available "To-Let" commercial properties that match the user's requirements directly on the map, allowing for seamless filtering and discovery.
    \item[\textbf{d.}] \textbf{Personalized Suitability Score Generation:} Upon selection of any point or property listing, calculate and display a quantitative "Personalized Suitability Score" that predicts the location's potential for success for the user's specific business profile.
    \item[\textbf{e.}] \textbf{Explainable AI (Prediction Breakdown):} Display a transparent breakdown of the key factors that positively and negatively influenced the suitability score, explaining the AI's reasoning to the user.
    \item[\textbf{f.}] \textbf{Trade Area Analysis:} Generate and visualize a realistic customer trade area based on a "5-minute walk" isochrone, providing an accurate view of the immediate customer base.
    \item[\textbf{g.}] \textbf{Detailed Competitor Analysis:} Display a list and map of nearby competitors, including their distance and calculated Popularity Score.
    \item[\textbf{h.}] \textbf{Location Comparison Dashboard:} Provide a feature for users to "pin" or save multiple locations and compare their key metrics side-by-side in a dedicated dashboard.
    \item[\textbf{i.}] \textbf{Automated PDF Report Generation:} Allow the user to download a comprehensive, professionally formatted PDF report summarizing the complete analysis for a chosen location with a single click.
\end{enumerate}

\subsection{Non-Functional Requirements}
The non-functional requirements define the quality attributes and operational standards of the system:

\begin{enumerate}
    \item[\textbf{a.}] \textbf{Accuracy:} The predictive model must achieve a satisfactory level of accuracy (measured by metrics like MAE  and R²) when evaluated against the test dataset to ensure trustworthy recommendations. %accuracy metrics might be RMSE or R², depending on the model's output type change it if needed.

    \item[\textbf{b.}] \textbf{Performance / Latency:} The system must return a full analysis (score, map data, etc.) within a few seconds (e.g., < 1 seconds) of a user selecting a location to ensure a responsive and fluid user experience.

    \item[\textbf{c.}] \textbf{Usability:} The user interface must be intuitive and easy to navigate for non-technical entrepreneurs, requiring minimal training to perform a full site analysis.

    \item[\textbf{d.}] \textbf{Reliability:} The web application must have high availability (e.g., 99.9\% uptime) and consistently provide predictions without crashing or producing critical errors.

    \item[\textbf{e.}] \textbf{Scalability:} The backend architecture must be able to handle multiple concurrent user requests without significant degradation in performance, ensuring the system remains responsive as the user base grows.

    \item[\textbf{f.}] \textbf{Security:} The admin interface and the underlying database must be protected against unauthorized access and modifications to ensure data integrity.

    \item[\textbf{g.}] \textbf{Maintainability:} The codebase for both the frontend and backend must be well-documented, modular, and adhere to best practices to facilitate future updates, bug fixes, and feature extensions.
\end{enumerate}

