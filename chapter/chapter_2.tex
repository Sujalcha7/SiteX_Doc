\chapter{Literature Review}

% \section{GIS-based Feature Engineering for Machine Learning}
\par A major challenge in spatial prediction tasks is transforming raw geographic data into structured features suitable for machine learning models. Al-Ruzouq et al. present a systematic GIS-based framework for generating predictive spatial features, commonly referred to as geospatial feature engineering \citep{alruzouq2019gis}. Their methodology involves the use of buffer analysis, proximity calculations, and spatial density measures to derive meaningful explanatory variables.

\par Specifically, the study demonstrates how GIS tools can be used to calculate distances to points of interest, count nearby competitors within defined radii, and measure road network density. These engineered features are then employed as inputs to machine learning models for land suitability prediction. This work is directly relevant to the present study, as it outlines a reproducible process for converting raw spatial data into structured inputs suitable for supervised learning-based site selection.

% \section{Applying Machine Learning to Retail Site Selection}
\par To overcome the limitations of traditional gravity-based models, researchers have increasingly adopted machine learning techniques for retail site selection. Machine learning models are capable of identifying complex, nonlinear relationships among multiple spatial and demographic variables without requiring explicit assumptions. Kuo et al. demonstrated the effectiveness of such approaches by applying decision tree models to predict the sales performance of new convenience stores \citep{kuo2002application}.

\par Their study incorporated demographic indicators, competitive intensity, and proximity-based location variables, showing that machine learning models significantly outperformed conventional statistical methods. The results confirmed that AI-based models can uncover interaction effects and hidden patterns that are not detectable using traditional analytical frameworks. This body of work provides strong empirical support for the use of supervised learning models in location suitability analysis.

% \section{GIS and Analytical Hierarchy Process for Site Location Decision Making}
\par The integration of Geographic Information Systems (GIS) with the Analytical Hierarchy Process (AHP) represents a well-established multi-criteria decision-making approach for site selection \citep{store2010retail}. The AHP framework decomposes the site selection problem into a hierarchy of criteria, subcriteria, and location alternatives, with stakeholders assigning weights through pairwise comparisons. GIS spatializes these criteria into raster or vector datasets, enabling geographic visualization and overlay analysis. In the context of a Predictive Site Selection System, the GIS-AHP methodology provides complementary insights to machine learning approacheswhile AHP captures expert knowledge and stakeholder preferences, machine learning models such as XGBoost learn from historical transaction data and competitive dynamics. Combining both approaches enables a hybrid decision support system that is both principled in treatment of subjective preferences and data-driven in treatment of empirical relationships.

% \section{Modern Location Intelligence Platforms}
\par The landscape of retail site selection has evolved significantly with the advent of "Location Intelligence" (LI) platforms. Unlike traditional desktop GIS, these modern tools often leverage cloud computing, big data from mobile devices, and machine learning to provide dynamic insights into consumer behavior and market potential. This section reviews key platforms that define the current state of the art.

% \subsection{Comprehensive GIS \& Market Analysis}
\par \textbf{Esri Business Analyst} remains the industry standard for traditional demographic and lifestyle analysis. It combines census data, consumer spending patterns, and business listings into a map-centric interface for trade area modeling \citep{pick_2020}. Its primary strength lies in its integration with the broader ArcGIS ecosystem, allowing organizations to perform complex suitability modeling and drive-time analysis using authoritative data sources \citep{esri_ba_2025}.

% \subsection{Human Mobility \& Foot Traffic Analytics}
\par A significant shift in site selection methodology is the move from static census data to dynamic human mobility data. Several platforms specialize in this domain:
\begin{itemize}
    \item \textbf{Placer.ai} has gained prominence for its user-friendly dashboard that visualizes foot traffic to any physical location. By aggregating mobile location signals, it allows retailers to understand "true" trade areas based on actual customer visits rather than theoretical radii \citep{placer_2025}.
    \item \textbf{SafeGraph} distinguishes itself as a data-first provider, focusing on high-precision Points of Interest (POI) and building geometry (polygons) rather than just centroids. It acts as a foundational layer for many data science teams building custom models \citep{safegraph_2024, kang_2020}.
    \item \textbf{Foursquare Places} leverages its history as a consumer app to provide a massive database of user check-ins and visits. Its "Shapes" and "Movement" datasets are widely used to validate site attendance and understand cross-shopping behaviors \citep{foursquare_2024}.
    \item \textbf{Near (formerly Gravy Analytics)} specializes in audience segmentation based on real-world behavior. Following its merger with Gravy Analytics, near offers high-fidelity insights into consumer profiles, such as identifying "luxury shoppers" or "commuters" based on their movement patterns over time \citep{near_2023}.
\end{itemize}

% \subsection{Spatial Data Science \& Visualization}
\par Platforms like \textbf{Carto} and \textbf{Mapbox} represent the intersection of developer tools and business intelligence. Carto focuses on "Spatial Data Science," enabling users to run SQL-based spatial queries directly on cloud data warehouses like BigQuery and Snowflake. It streamlines the ingestion of third-party datasets for site suitability scoring \citep{carto_2024}. Similarly, Mapbox, while primarily a mapping engine, offers robust analytics layers for traffic density and movement, helping retailers assess the visibility and accessibility of street segments \citep{mapbox_2024}.

% \subsection{AI-Driven Predictive Modeling}
\par \textbf{SiteZeus} exemplifies the shift towards prescriptive analytics. Instead of just visualizing data, it employs machine learning to build revenue forecasting models for multi-unit brands. By correlating site attributes with existing sales performance, it predicts the potential revenue of new locations with high granularity \citep{sitezeus_2024}.


% \begin{table}[ht]
%     \centering
%     \caption{Summary of Key Literature and Platforms in Site Selection}
%     \label{tab:lit_review_summary}
%     \begin{tabular}{|p{0.25\textwidth}|p{0.35\textwidth}|p{0.3\textwidth}|}
%     \hline
%     \textbf{Study / Platform} & \textbf{Key Focus} & \textbf{Key Findings / Contribution} \\
%     \hline
%     \citet{alruzouq2019gis} & GIS-based Feature Engineering & Framework for transforming raw spatial data (buffers, density) into ML features. \\
%     \hline
%     \citet{kuo2002application} & Machine Learning (Decision Trees) & ML models outperform traditional statistics by capturing non-linear relationships. \\
%     \hline
%     \citet{store2010retail} & GIS + Analytical Hierarchy Process (AHP) & Integrates expert subjectivity (AHP) with objective spatial data (GIS). \\
%     \hline
%     \citet{pick_2020} (Esri) & Standard Demographic Analysis & Established suitability modeling using census and lifestyle data. \\
%     \hline
%     Placer.ai \citep{placer_2025} & Human Mobility Analytics & Defines "true" trade areas using dynamic mobile signal data. \\
%     \hline
%     SiteZeus \citep{sitezeus_2024} & AI-Driven Revenue Forecasting & Prescriptive analytics to predict specific revenue figures for new sites. \\
%     \hline
%     \end{tabular}
% \end{table}