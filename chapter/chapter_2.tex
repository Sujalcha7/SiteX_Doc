\chapter{Literature Review}

\section{GIS-based Feature Engineering for Machine Learning}
A major challenge in spatial prediction tasks is transforming raw geographic data into structured features suitable for machine learning models. Al-Ruzouq et al. present a systematic GIS-based framework for generating predictive spatial features, commonly referred to as geospatial feature engineering \citep{alruzouq2019gis}. Their methodology involves the use of buffer analysis, proximity calculations, and spatial density measures to derive meaningful explanatory variables.

Specifically, the study demonstrates how GIS tools can be used to calculate distances to points of interest, count nearby competitors within defined radii, and measure road network density. These engineered features are then employed as inputs to machine learning models for land suitability prediction. This work is directly relevant to the present study, as it outlines a reproducible process for converting raw spatial data into structured inputs suitable for supervised learning-based site selection.

\section{The Huff Gravity Model of Retail Location}
David Huff’s seminal work established one of the earliest quantitative frameworks for retail site evaluation, providing a mathematical basis for estimating consumer choice behavior \citep{huff1963mathematical}. The Huff Gravity Model proposes that the probability of a consumer patronizing a specific retail outlet depends on the outlet’s relative attractiveness and the impedance of travel, typically expressed through distance or travel time. Attractiveness is modeled using proxy variables such as store size, brand appeal, and service quality, while impedance represents the cost of accessing the location.

Although the model has been widely applied in retail geography and spatial analysis, its explanatory power is constrained by strong simplifying assumptions. The reliance on predefined functional forms limits its ability to capture nonlinear interactions and complex spatial dynamics. Factors such as competitive overlap, micro-level demographic clustering, and urban visibility effects are not explicitly modeled and instead remain implicit. These limitations have motivated the exploration of data-driven approaches that can learn location–performance relationships directly from observed data.

\section{Applying Machine Learning to Retail Site Selection}
To overcome the limitations of traditional gravity-based models, researchers have increasingly adopted machine learning techniques for retail site selection. Machine learning models are capable of identifying complex, nonlinear relationships among multiple spatial and demographic variables without requiring explicit assumptions. Kuo et al. demonstrated the effectiveness of such approaches by applying decision tree models to predict the sales performance of new convenience stores \citep{kuo2002application}.

Their study incorporated demographic indicators, competitive intensity, and proximity-based location variables, showing that machine learning models significantly outperformed conventional statistical methods. The results confirmed that AI-based models can uncover interaction effects and hidden patterns that are not detectable using traditional analytical frameworks. This body of work provides strong empirical support for the use of supervised learning models in location suitability analysis.

% \section{The Role of Demographics in Site Selection}
% Existing literature consistently emphasizes that physical location characteristics alone are insufficient for predicting retail success. Demographic and socioeconomic attributes of the surrounding population play a crucial role in shaping consumer demand. Variables such as population density, household income, age distribution, and lifestyle indicators have been shown to strongly influence retail performance and long-term viability.
% Hernández and Bennison provide a comprehensive review of demographic factors affecting retail location decisions, highlighting the importance of aligning site selection with target customer profiles \citep{hernandez2007art}. Their analysis underscores that successful retailers tailor their location strategies to match the preferences and behaviors of local consumer segments. Integrating demographic data into predictive models enhances their accuracy and practical relevance for site selection.



\section{GIS and Analytical Hierarchy Process for Site Location Decision Making}
The integration of Geographic Information Systems (GIS) with the Analytical Hierarchy Process (AHP) represents a well-established multi-criteria decision-making approach for site selection \citep{store2010retail}. The AHP framework decomposes the site selection problem into a hierarchy of criteria, subcriteria, and location alternatives, with stakeholders assigning weights through pairwise comparisons. GIS spatializes these criteria into raster or vector datasets, enabling geographic visualization and overlay analysis. In the context of a Predictive Site Selection System, the GIS-AHP methodology provides complementary insights to machine learning approaches—while AHP captures expert knowledge and stakeholder preferences, machine learning models such as XGBoost learn from historical transaction data and competitive dynamics. Combining both approaches enables a hybrid decision support system that is both principled in treatment of subjective preferences and data-driven in treatment of empirical relationships.
