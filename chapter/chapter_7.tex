\chapter{EXPECTED RESULT}\label{chap:outcomes}
\section{Expected Result}

When we finish our final year project, we don’t just want a grade. We want to have built something that actually works and means something. Our main goal is to create a real, live web application that anyone can use. Think of a friend, a dai or didi, wanting to open a small cafe in Bhaktapur. We want them to be able to use our tool to move past just a 'gut feeling' and make a decision based on real data. It’s about giving them a better, safer chance to succeed.

\vspace{0.4cm}
\noindent
This tool won’t just spit out a random 'success score.' We expect it to give a clear breakdown, explaining 'why' a spot is good. Maybe it’s close to a college, or maybe there aren't too many other cafes nearby. The idea is to give a small business owner the kind of smart insights that big companies have, but for our own local streets. We want to level the playing field, so that starting a business feels less like a gamble.

\vspace{0.4cm}
\noindent
On a personal level, this project is everything for us. It’s our chance to take all the theory we’ve learned over seven semesters—all the GIS, the data engineering, the machine learning with XGBoost—and build something from scratch. We’re creating our own dataset, training our own model, and deploying a full application. This is the real test. We expect to come out of this not just as students, but as engineers who have actually built a complete, end-to-end data science solution for a problem we genuinely care about, right here in Nepal.
