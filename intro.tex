\pagenumbering{roman}
% 		\large
% 			\chapter*{BONAFIDE CERTIFICATE}
% 		\normalsize 
% 		\addcontentsline{toc}{section}{BONAFIDE CERTIFICATE}

% 	 This is to certify that the project entitled \textbf{“Voice Assistant for Visually Impaired People”} submitted by \textbf{ Avyudaya Karn (590301), Birendra Chawal (590307), Rohit Kumar Mahato (590316) and  Sunil Chikanbanjar (590336)} in partial fulfillment of the requirements for the award of the degree of \textbf{Bachelor of Engineering in Computer Engineering} of Purbanchal University, is a bonafide work to the best of our knowledge and may be placed before the examination board for their consideration.\\ 

%  \vspace{1 cm}

% \underline{\textbf{Panel Of Examiners:}}
% \qquad\qquad\qquad\qquad\qquad\qquad\qquad\qquad\underline{\textbf{Signature}}

% \vspace{1cm}

% \textbf{1. External Examiner} \\
% Person 1 Name \\
% Designation \\
% Institute/University Name \\
% Date\\

% \vspace{1cm}

% \textbf{2. Supervisor} \\
%  Person 2 Name  \\
%  Designation  \\
%  Institute/University Name  \\
%  Date \\

% \textbf{3. Head of Department} \\
% Person 3 Name \\
% HoD \\
% Khwopa Engineering College \\
% Date\\

% \vspace{1cm}





% 		\large
% 		\chapter*{Copyright}
% 		\normalsize
% 		\addcontentsline{toc}{section}{Copyright}
% 			The author has agreed that the library, Khwopa   Engineering College  may make this report freely available for inspection. Moreover, the author has agreed that permission for the extensive copying of this project report for scholarly purpose may be granted by supervisor who supervised the project work recorded here in or, in absence the Head of the Department where in the project report was done. It is understood that the recognition will be given to the author of the report and to Department of Computer Engineering, KhEC in any use of the material of this project report. Copying or publication or other use of this report for financial gain without approval of the department and author’s written permission is prohibited. Request for the permission to copy or to make any other use of material in this report in whole or in part should be addressed to: \\
% 			\vspace{1cm} \\
% 			Head of Department \\
% 			Department of Computer Engineering\\
% 			Khwopa Engineering College\\
% 			Libali-08,\\
% 			Bhaktapur, Nepal\\
% 		\pagebreak


\large
\chapter*{Acknowledgement}
\normalsize
\addcontentsline{toc}{section}{Acknowledgement}
We take this opportunity to express our deepest and sincere gratitude for the chance to work on our project, \textbf{‘Predictive Site Selection System}. This work is submitted as a partial fulfillment of the requirements for the seventh semester of our Bachelor of Computer Engineering degree under Purbanchal University.

\noindent
We are sincerely thankful to our institution, \textbf{Khwopa Engineering College}, for providing us with the platform and resources necessary to undertake this project. We would like to extend our special and heartfelt gratitude to our Head of the Department of Computer Engineering, \textbf{Er. Bikash Chawal}, for his invaluable guidance and for providing us with this golden opportunity to apply our skills to a real-world challenge.

\noindent
This project is not just an academic exercise; it is our endeavor to bridge the gap between complex data science and practical, everyday business decisions. We would also like to express our sincere gratitude to all those whose ideas, insights, and innovations have inspired this project.
\begin{table}[h]
	\begin{tabular}{ll}
		Anish Maka      & (780304) \\
		Roj Gosai    & (780334) \\
		Subekshya Kadel & (780341) \\
		Sujal Koju & (780342)
	\end{tabular}
\end{table}
\pagebreak

\large
\chapter*{Abstract}
\normalsize
\addcontentsline{toc}{section}{Abstract}
Selecting the optimal location for a new retail business is a critical factor that significantly influences its success. Traditional methods for site selection often rely on manual surveys, intuition, and static demographic data, which fail to account for dynamic urban patterns and competitor influence. This project proposes an AI-powered tool that leverages geospatial data, competitor analysis, and crowd origin sources—such as schools, parks, and public places—to predict the success rate of a shop at a specific location. By integrating machine learning models with interactive mapping systems, the tool analyzes factors like monthly customer footfall, competitor popularity, and proximity to high-traffic areas to provide accurate predictions and actionable recommendations. The system also features a success heatmap visualization, enabling users to explore and compare potential locations effectively. Designed to support small and medium businesses, this smart site selection tool democratizes access to advanced location intelligence for informed retail planning. a reliable and effective solution.\\\\
\textbf{Keywords}:\textit{Retail Site Selection, Geospatial Analysis, Crowd Origin Mapping, Interactive Map Visualization}

\pagebreak